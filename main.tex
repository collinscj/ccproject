\documentclass{article}
\usepackage[utf8]{inputenc}

\title{ccproject}
\author{cjc }
\date{October 2018}

\usepackage{natbib}
\usepackage{graphicx}

\begin{document}

\maketitle

\section{Systematic Errors In A Homodyne Interferometer}

We will consider the effect of systematic errors in a homodyne interferometer
which are periodic over one fringe, and how they translate into noise
given a particular type of signal. There are two stages to this analysis.
First, we look a the difference between the actual phase corresponding
to the position of a target mirror and the measured phase as extracted
from the fringe unwrapping function. We calculate the Fourier Series
of the derivate of the measured phase w.r.t. the actual phase. It
is found that, for some important and feasible types of imperfections
in the interferometer, this series can be calculated analytically.
Next, we look at how these periodic spurious extra signals will translate
into apparent velocity or displacement noise as a function of frequency,
given a particular type of motion of the target mirror.




\section{Conclusion}
``I always thought something was fundamentally wrong with the universe'' \citep{adams1995hitchhiker}

\bibliographystyle{plain}
\bibliography{references}
\end{document}

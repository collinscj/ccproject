\documentclass{article}
\usepackage[utf8]{inputenc}

\title{ccproject}
\author{cjc }
\date{October 2018}

\usepackage{natbib}
\usepackage{graphicx}

\begin{document}

\maketitle

\section{Systematic Errors In A Homodyne Interferometer}

We will consider the effect of systematic errors in a homodyne interferometer
which are periodic over one fringe, and how they translate into noise
given a particular type of signal. There are two stages to this analysis.
First, we look a the difference between the actual phase corresponding
to the position of a target mirror and the measured phase as extracted
from the fringe unwrapping function. We calculate the Fourier Series
of the derivate of the measured phase w.r.t. the actual phase. It
is found that, for some important and feasible types of imperfections
in the interferometer, this series can be calculated analytically.
Next, we look at how these periodic spurious extra signals will translate
into apparent velocity or displacement noise as a function of frequency,
given a particular type of motion of the target mirror.

\subsection{Fourier Analysis of periodic systematic errors}

As discussed above, the phase is reconstructed from the measured photodiode
signals by taking the four-quadrant ratio

\begin{equation}
R\equiv ArcTan2\left(PD2-PD1,PD2-PD3\right)
\end{equation}


In the case of a perfect interferometer, this would be equal to

\begin{equation}
R_{ideal}\equiv\frac{Sin\left(\phi\right)-Cos\left(\phi\right)}{Sin\left(\phi\right)+Cos\left(\phi\right)}=\frac{Sin\left(\phi-\frac{\pi}{4}\right)}{Cos\left(\phi-\frac{\pi}{4}\right)}
\end{equation}


In this case the measured phase, $$\phi_{m}$$ can be extracted as

\begin{equation}
\phi_{m}=ArcTan\left(R_{ideal}\right)=\phi-\frac{\pi}{4}
\end{equation}


Apart from an unimportant and known offset, the phase is recovered.

If one or more of the photodiode signals are not of their ideal form,
this will lead to a difference between $\phi$ and $\phi_{measured}$.
To model the most general case would require a numerical model. However,
we can derive analytically the systematic error in the signal produced
by certain types of imperfection in the interferometer. We will look
at a small number of representative examples of these effects. We
will look at three different effects. First we will consider the case
of the phase measured at one of the photodiodes having a constant
offset. This would result from an imperfect wave plate, or unwanted
birefringence in another component. Next, we consider the effect of
adding a constant power offset to one of the signals; this may occur
if there were some component of non-interfering light on the PD, or
an unwanted DC offset in the photodiode circuit. Finally, we will
linearly rescale one of the photodiode signals. This represents the
effect of one PD seeing a different amount of interfering light from
the others


\subsubsection{Phase offset in one photodiode}

We model this by adding an offset, $d$, to the 'Sine' photodiode
signal. $\phi_{measured}$ will then become

\begin{equation}
\phi_{measured}=ArcTan\left(\frac{Sin\left(\phi+d\right)-Cos\left(\phi\right)}{Sin\left(\phi+d\right)+Cos\left(\phi\right)}\right)
\end{equation}


If we assume $d$ is small $\left(d\ll1\right)$, we can expand this
in a Taylor series in . It is more transparent to expand the derivative
of this w.r.t. $\phi$ (which is 1 in the ideal case). This gives
us:

\begin{equation}
\frac{d\phi_{measured}}{d\phi}=1-2Sin\left(\phi\right)Cos\left(\phi\right)d+\left(\frac{1}{2}Sin^{4}\left(\phi\right)+3Sin^{2}\left(\phi\right)Cos^{2}\left(\phi\right)+\frac{3}{2}Cos^{4}\left(\phi\right)\right)d^{2}+o\left(d^{3}\right)
\end{equation}


It is found that the terms of order $d^{n}$ have coefficients of
the form $Sin^{a}\left(\phi\right)Cos^{2n-a}\left(\phi\right)$.  This
will produce spurious signals at high spatial frequencies, but if
$d$ is small these will quickly decrease. It will be convenient to
express this in terms of the form $Sin\left(n\phi\right)$ and $Cos\left(n\phi\right)$,
so that the Fourier Transform w.r.t. $\phi$ can be easily found.
The coefficient of a given Fourier component in $\phi$ will be an
infinite series in $d$.

It is found that only the even order Fourier Coefficients of $\frac{d\phi_{measured}}{d\phi}$
are non-zero. These are given by

\begin{equation}
b_{2n}=\frac{e^{id}}{4}a_{2n-2}+Cos(d)a_{2n}+\frac{e^{-id}}{4}a_{2n+2}
\end{equation}


where the $a_{n}$ are given by

\begin{equation}
a_{2n}=\left(-\frac{1}{4}\right)^{n}\left(1-e^{2id}\right)^{n}\,_{2}F_{1}\left(\frac{n+1}{2},\frac{n+2}{2},n+1;\left(-4Sin^{2}(d)Cos^{2}(d)\right)\right)
\end{equation}


For $d\ll1$ this gives the result

\begin{equation}
b_{2m}\approx m\left(\frac{id}{2}\right)^{m}
\end{equation}



\subsubsection{Power offset on one photodiode}

A similar result is observed if the offset terms of the photodiode
signals do not exactly cancel out. We will examine the case of an
offset power, $P$, on one of the photodioes. If we add this constant
power to the Sine Photodiode, we have

\begin{equation}
\phi_{measured}=ArcTan\left(\frac{P+Sin\left(\phi\right)-Cos\left(\phi\right)}{P+Sin\left(\phi\right)+Cos\left(\phi\right)}\right)
\end{equation}


Proceeding as before to take the derivative of this w.r.t. $\phi$,
we find

\[
\frac{d\phi_{measured}}{d\phi}=\frac{1+Psin(\phi)}{1+2Psin(\phi)+P^{2}}=(1+Psin(\phi))\left(1+{\displaystyle \sum_{n=1}^{\infty}\left(iPe^{i\phi}-iPe^{-i\phi}-P^{2}\right)^{n}}\right)
\]


We can again take the Fourier Transform, finding that the Fourier
Coefficients $b_{m}$ for $\frac{d\phi_{measured}}{d\phi}$ will be
equal to

\begin{equation}
b_{m}=\frac{P}{2i}a_{m-1}+a_{m}-\frac{P}{2i}a_{m+1}
\end{equation}


where the $a_{n}$ are given by

\[
a_{n}=\sum_{j=1}^{\infty}\left((iP)^{n}P^{2}(-P^{2})^{2j}\frac{(1+2j+n)!}{(1+2j)!n!}\,_{2}F_{1}\left(-\frac{1}{2}-j,-j,n+1;\frac{4}{P^{2}}\right)\right)
\]


\begin{equation}
+\sum_{j=1}^{\infty}\left((iP)^{n}(-P^{2})^{2j}\frac{(2j+n)!}{(2j)!n!}\,_{2}F_{1}\left(\frac{1}{2}-j,-j,n+1;\frac{4}{P^{2}}\right)\right)
\end{equation}


For small $P$, these coefficients will be of approximately equal
to

\begin{equation}
b_{m}\approx\frac{1}{2}\left(iP\right)^{m}
\end{equation}



\subsubsection{Difference in fringe visibility on one photodiode}

Finally, we will look at the case of one photodiode having a different
visibility to the others ($i.e.$ the Sine or Cosine term being multiplied
by a constant other than one). The ellipse-fitting algorithm during
the calibration should prevent this situation from arising, but it
could result from either the optical visibilty changing, due to a
large motion of the test-mass, or the gain of one or more of the photodetector
circuits, changing in between calibrations. As with the previous calculations,
we start by putting a systematic error onto the Sine Photodiode, then
proceed to differentiate the unwrapped signal and then to look at
the behaviour of its Fourier Coefficients. We rescale the Sine signal
by a factor of $1+k$ to get:

\begin{equation}
\phi_{measured}=ArcTan\left(\frac{(1+k)Sin\left(\phi\right)-Cos\left(\phi\right)}{(1+k)Sin\left(\phi\right)+Cos\left(\phi\right)}\right)
\end{equation}


We then find

\begin{equation}
\frac{d\phi_{measured}}{d\phi}=\frac{1+k}{cos^{2}(\phi)+(1+k)^{2}sin^{2}(\phi)}=\frac{1+k}{1+(2k+k^{2})sin^{2}(\phi)}
\end{equation}


The Fourier Coefficients of this are found to be

\[
b_{n}=\sum_{j=1}^{\infty}(1+k)(k(2+k))^{2j+n+1}2^{-2n-4j-1}\frac{(2j+n+1)!}{(2j+1)!n!}\,_{2}F_{1}\left(-\frac{1}{2}-j,-j,n+1;16\right)
\]


\begin{equation}
+\sum_{j=1}^{\infty}(1+k)(k(2+k))^{2j+n}2^{-2n-4j-1}\frac{(2j+n+1)!}{(2j+1)!n!}\,_{2}F_{1}\left(\frac{1}{2}-j,-j,n+1;16\right)
\end{equation}


We once again find, for small values of $k$, an exponential decay
of the $a_{n}$ with $n$, with coefficients that depend on $k$.
For $k<0.1$ the behaviour is well approximated by

\begin{equation}
\Vert a_{n}\Vert\simeq0.204k^{2}e^{(-2.99-0.033/k)n}
\end{equation}


Similar analyses can be performed for other simple types of errors
in the photodiode signals, as well as for combinations of these errors,
but the algebra quickly becomes onerous, and numerical methods are
preferable.


\subsection{Noise spectrum produced by systematic errors given a finite amplitude
motion}

The effect of these systematic errors on the observed noise spectrum
will depend on the speed and amplitude of the actual motion being
measured. For simplicity, we consider only components of systematic
errors that are periodic over one wavelength. Systematic errors at
spatial frequencies longer than a wavelength will in general produce
low frequency drifts in the signal which are easy to account for,
for instance by recalibrating at different ranges. We have seen that
the measured phase will be equal to the actual phase, plus an infinite
series of (eventually decreasing) terms with spatial frequencies that
are integer multiples of $\frac{1}{\lambda}$, where $\lambda$ is
the wavelength of light in the interferometer. If the target mirror
does not move, these systematic errors might produce a DC error on
the measurement of the position, but there will be no noise added
at higher frequencies. We will next examine the case where the target
mirror travels at a constant velocity, $V$. Then the actual phase
as a function of time will be given by

\[
\phi\left(t\right)=\frac{2Vt}{\lambda}
\]


The measured phase as a function of time will then satisfy:

\[
\frac{d\phi_{measured}}{dt}=\frac{d\phi_{measured}}{d\phi}\frac{d\phi}{dt}=\frac{d\phi_{measured}}{d\phi}\frac{2V}{\lambda}=\frac{2V}{\lambda}\left(\sum_{m=-\infty}^{\infty}a_{m}Exp\left(im\frac{4Vt}{\lambda}\right)\right)
\]
We can then determine the Fourier Transform of $\phi_{measured}$:

\[
FT\left(\phi_{measured}\right)=\frac{1}{i\omega}FT\left(\frac{d\phi_{measured}}{dt}\right)=\frac{1}{i\omega}\frac{2V}{\lambda}\left(\sum_{m=-\infty}^{\infty}a_{m}\delta(\omega-m\frac{4Vt}{\lambda})\right)
\]


As expected, the Fourier Transform in the time domain is once again
a sum of delta functions, rescaled by a velocity-dependent factor.

In the more general case that the test-mass does not simply move with
a constant velocity, we expect to find that the Fourier peaks in $\frac{d\phi_{measured}}{d\phi}$
become spread out as well as rescaled, in ways dependent on the functional
form of $\phi(t)$. We will look at a few particular cases and make
some general comments, which may help provide some interpretation
and context to the results of numerical simulations in the following
section.

We will look at perhaps the next simplest case, in which the displacement
of the test mass varies sinusoidally with time,

\begin{equation}
\phi\left(t\right)=ASin\left(\omega_{0}t+\psi\right)
\end{equation}


Where A and $\omega_{0}$ are constants. We split these into 3 sub-cases
according to the amplitude of the motion:

i. Motion much less than one fringe,

$A\ll2\pi$ or $B\ll2\pi$ .

ii. Motion of order one fringe,

$A\approx2\pi$ or $B\approx2\pi$ . And,

iii. Motion much greater than one fringe,

$A\gg2\pi$ or $B\gg2\pi$ .


\subsubsection{Sinusoidal motion much less than one fringe.}

Putting $\phi\left(t\right)=ASin\left(\omega_{0}t+\psi\right)$ with
$A\ll1$, we will have motion over only a small part of a fringe.
For simplicity, we will at present assume $\psi=0;$ more general
values will not significantly affect the following analysis. Qualitatively,
what we expect to find is that the Fourier Components in the fringe
unwrapping function of order much less than $\frac{1}{A}$ will contribute
an additional term at the frequency of the motion itself. Fourier
Components of order close to $\frac{1}{A}$ may produce aliasing effects,
while components of order much greater than $\frac{1}{A}$ will directly
contribute high frequency noise. The assumption that $A\ll1$ means
that these last contributions will be negligible. as can be seen by
expanding the Fourier transform as a series

\[
FT\left(\frac{d\phi_{measured}}{dt}\right)(\omega)=\intop_{-T/2}^{T/2}\left(e^{-i\omega t}\left(\sum a_{n}e^{inAsin\left(\omega_{0}t\right)}\right)\right)dt
\]


Here $T$ is the measurement time, which is assumed to be large. For
motion much less than one fringe we can expand the outer exponential,
to give

\[
FT\left(\frac{d\phi_{measured}}{dt}\right)(\omega)=\intop_{-T/2}^{T/2}\left(e^{-i\omega t}\left(\sum a_{n}(1+inAexp\left(i\omega_{0}t\right)-n^{2}A^{2}exp(2i\omega_{0}t)+...)\right)\right)dt
\]


\[
\approx\sum(a_{n}niA\delta\left(\omega-\omega_{0}\right)-a_{n}n^{2}A^{2}\delta\left(\omega-2\omega_{0}\right)+...)
\]


As expected, we observe a signal at the frequency of the motion, as
well as at integer multiples of that frequency, with each higher frequency
term suppressed by a factor of $A$, if $a_{1}$ is non-zero, or of
$2A$ is $a_{2}$ is the first non-zero term. These delta functions
would become spread out first of all due to the finite time of the
measurement. Additionally, in any real case, the sinusoidal functional
form of the displacement as a function of time will often be an approximation;
for instance for a mechanical oscillator, there will likely be phase
noise due to a non-zero damping term.


\subsubsection{Sinusoidal motion of order one fringe}

In the specific case that the motion is over exactly one fringe, we
can give an exact analytical result. If we assume for the sake of
simplicity that the measurement time $T$ is some integer multiple,
$N$ , of the period of the motion, then we find

\[
FT\left(\frac{d\phi_{measured}}{dt}\right)(\omega)=\intop_{-N\pi/\omega_{0}}^{N\pi/\omega_{0}}\left(e^{-i\omega t}\left(\sum a_{n}e^{inAsin\left(\omega_{0}t\right)}\right)\right)dt
\]


\[
=N\sum na_{n}J_{0}(nA\frac{\omega}{\omega_{0}})+
\]


More generally, we can split the motion into two sections, one corresponding
to the motion over one fringe, and the other an additional term due
to motion over less than one fringe. The first part will give a result
close to that for motion over exactly one fringe, while the additional
part will be close to the result of case $i$ above.


\subsubsection{Sinusoidal motion much greater than one fringe.}

Putting $\phi\left(t\right)=ASin\left(\omega_{0}t+\psi\right)$ with
$A\gg1$, we expect all Fourier Components to contribute to the observed
signal, and they should again contribute without any additional weighting
factor. This is similar to the case of motion with a constant velocity,
but in this case the velocity itself varies sinusoidally in the range
$-A\omega_{0}\leq v\leq A\omega_{0}$. This range will not be evenly
sampled; the density at a given value of velcoity, $v_{0}$ will be
given by $\rho(v_{0})=\left(\frac{d}{d\phi}\left(\frac{d\phi}{dt}\right)\right)^{-1}$.
We can split the Fourier Transform integral into two parts, one of
which is the contribution from the complete traversed fringes, of
which we assume there is some number, N, and another contribution
from the 'remainder' term.

The peaks in the power spectrum of $\phi_{m}\left(\phi\right)$ will
be broadened to create a 'shelf' in frequency space. The right-hand
edge of this shelf will be at the frequency corresponding to the highest
velocity of the test-mass, $i.e.$ at angular frequency $\omega_{m}=\frac{4mV}{\lambda}=\frac{4mA\omega_{0}}{\lambda}$,
and will extend down to zero frequency. The heights of successive
shelves will be determined by the coefficients $b_{m}$ calculated
above.

We can make a stronger statement about the expected shape and height
of this shelf in frequency space. If the 




.. \citep{adams1995hitchhiker}

\bibliographystyle{plain}
\bibliography{references}
\end{document}
